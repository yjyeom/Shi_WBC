\documentclass{ieeeaccess}
\usepackage{cite}
\usepackage{amsmath,amssymb,amsfonts}
%\usepackage{algorithmic}
\usepackage{graphicx}
\usepackage{textcomp}

\usepackage{thmtools}
\declaretheorem[numberwithin=section]{theorem}
\declaretheorem[numberwithin=section]{lemma}
\newcommand{\rank}{\mathop{\mathrm{rank}}}



\usepackage{algpseudocode}
\usepackage{algorithm}
\usepackage{mathtools}
\usepackage{setspace}
%\usepackage{algorithmicx}
%\usepackage{algorithm2e}

\newcommand{\multiline}[1]{%
    \begin{tabularx}{\dimexpr\linewidth-\ALG@thistlm}[t]{@{}X@{}}
        #1
    \end{tabularx}
}

\newcommand{\Input}[1]{\algrenewcommand{\alglinenumber}[1]{Input: \ \setcounter{ALG@line}{\numexpr##1-1}} #1}
\newcommand{\Step}[1]{\algrenewcommand{\alglinenumber}[1]{Step ##1: } #1}
\newcommand{\NoNumber}{\algrenewcommand{\alglinenumber}[1]{\setcounter{ALG@line}{\numexpr##1-1} \ \ \ \ \ \ \ \ \ \ }}
\newcommand{\Output}[1]{\algrenewcommand{\alglinenumber}[1]{Output:\setcounter{ALG@line}{\numexpr##1-1}} #1}

\makeatletter
\newenvironment{breakablealgorithm}
  {% \begin{breakablealgorithm}
   \begin{flushleft}
     \refstepcounter{algorithm}% New algorithm
     \hrule height.8pt depth0pt \kern2pt% \@fs@pre for \@fs@ruled
     \renewcommand{\caption}[2][\relax]{% Make a new \caption
       {\raggedright\textbf{\fname@algorithm~\thealgorithm} ##2\par}%
       \ifx\relax##1\relax % #1 is \relax
         \addcontentsline{loa}{algorithm}{\protect\numberline{\thealgorithm}##2}%
       \else % #1 is not \relax
         \addcontentsline{loa}{algorithm}{\protect\numberline{\thealgorithm}##1}%
       \fi
       \kern2pt\hrule\kern2pt
     }
  }{% \end{breakablealgorithm}
     \kern2pt\hrule\relax% \@fs@post for \@fs@ruled
   \end{flushleft}
  }
\makeatother


\def\BibTeX{{\rm B\kern-.05em{\sc i\kern-.025em b}\kern-.08em
    T\kern-.1667em\lower.7ex\hbox{E}\kern-.125emX}}
\begin{document}
\history{Date of publication xxxx 00, 0000, date of current version xxxx 00, 0000.}
\doi{10.1109/ACCESS.2017.DOI}

\title{Cryptanalysis of Shi's White-box Encryption Scheme}
\author{\uppercase{Hyoungshin Yim}\authorrefmark{1},
\uppercase{Yongjin Yeom\authorrefmark{1,2}, and Ju-Sung Kang
}.\authorrefmark{1,2}
}
\address[1]{Department of Financial information security, 
Kookmin University, Seoul 02707, South Korea}
\address[2]{Department of Information Security, Cryptology, and Mathematics
Kookmin University, Seoul 02707, South Korea}

\tfootnote{
This work has supported 
by the National Research Foundation of Korea (NRF) grant funded 
by the Korea government (MSIT) (NO. 2021M1A2A2043893)}


\markboth
{Hyungshin Yim \headeretal: Preparation of Papers for IEEE TRANSACTIONS and JOURNALS}
{Hyungshin Yim \headeretal: Preparation of Papers for IEEE TRANSACTIONS and JOURNALS}

\corresp{Corresponding author: Yongjin Yeom (e-mail: salt@kookmin.ac.kr).}


\begin{abstract}
Structural analysis is the study of finding component functions for a given function.  
In this paper, we proceed with structural analysis of structures consisting of the S (nonlinear Substitution) layer and the A (Affine or linear) layer. 
Our main interest is the $S^{(2)}\circ A\circ S^{(1)}$ structure with different substitution layers and large input/output sizes. 
The purpose of our structural analysis is to find the functionally equivalent oracle $F^*$ 
and its component functions for a given encryption oracle $F=S^{(2)}\circ A\circ S^{(1)}$. 
As a result, we can construct the decryption oracle ${F^*}^{-1}$ explicitly and 
break the one-wayness of the building blocks used in a White-box implementation. 
Our attack consists of two steps: S layer recovery using multiset properties and A layer recovery using differential properties. 
We present the attack algorithm for each step and estimate the time complexity. 
Finally, we discuss the applicability of $S^{(2)}\circ A\circ S^{(1)}$ structural analysis in a White-box Cryptography environment.
\end{abstract}

\begin{keywords}
Cryptanalysis, Structural analysis, White-box cryptography, White-box Implementation
\end{keywords}

\titlepgskip=-15pt

\maketitle

\section{Introduction}
\label{sec:introduction}
\PARstart{C}{ryptographic} 
technology is widely used in information and communication services for data protection and authentication. 
In encryption technology, encryption keys are essential for data and information and communication services authentication. 

\begin{itemize}
\item Intro. to WBC
\item Shi's model
\item Structural analysis
\item Our contribution
\end{itemize}
The security of the cryptosystem can be guaranteed only when the encryption key is safely protected from various attackers. The attacker models that threaten the security of cryptosystems include black-box attacks, gray-box attacks, and white-box attacks. The black-box attack is carried out through input and output values in unknown assumptions inside the cryptosystem. The gray-box attack is a technique that acquires and attacks side-channel information such as a cryptographic module's power and electromagnetic waves. Among them, the white-box attack assumes the most potent attacker. The white-box attack is a model in which an attacker takes control of the cryptosystem and neutralizes the cryptography. For example, there are dump and change of memory or register, monitoring the execution process, and the like. This has attracted attention to protect encryption keys used for copyright protection from exposure in media players and set-top boxes. Currently, the scope of use is expanding due to the safe execution of financial applications in a mobile environment and the prevention of firmware forgery in embedded devices [1]. In 2002, Chow et al [2, 3]. suggested the possibility of white-box cryptography of AES (Advanced Encryption Standard) and DES (Data Encryption Standard) along with the concept of white-box attackers, and various white-box cryptography technologies were proposed after that. The security of white-box cryptography generally aims at all or part of preventing exposure to encryption keys, one-wayness of encryption or decryption, and preventing the reproduction of cryptosystems [4]. It was analyzed that most of the white-box cryptography designed in a table reference method, including white-box cryptography such as Chow, does not satisfy any security goals. In general, one-wayness, the security of the white-box, cannot be maintained based on the analysis results of the SASAS structure consisting of the non-linear function S-box and the affine function proposed by Biruykov et al. [6, 7]. The analysis results of structures other than the SASAS structure are also the same. However, various attempts are still underway, and white-box cryptographic products that adopt undisclosed techniques are also actively spreading [5].
This paper proposes an attack method on the light-weight white-box encryption scheme for securing distributed embedded devices presented by Shi et al. to IEEE Transaction on Computers in 2019 [8]. The LW-WBC (Light-Weight White-Box Cryptography) proposed by Shi et al. has a Feistel structure and protects the input and output value of the XOR (exclusive or) operation by using a table reference method in each round. In addition, it was argued that it was safe against the existing white-box attack method. The LW-WBC, a 60-bit $S^{(1)} AS^{(2)}$ structure with different nonlinear S-box sizes, increases attack complexity and enhances security by not exposing the linear functions used. However, as a result of applying Biruykov's SASAS structural analysis, it was confirmed that the inverse functions of each round could be efficiently obtained [8]. This cannot guarantee one-wayness, which is the security of the white-box cryptography. 
This paper presents the existing white-box cryptographic model and structural analysis studies. First, we implement LW-WBC based on C language and analyze the security evidence claimed by Shi. After that, we present the structural analysis algorithm of the $S^{(1)} AS^{(2)}$ structure and calculate the attack complexity based on it. Finally, the attack is carried out by applying structural analysis to the LW-WBC using Python language. The results of this study can be used in commercialized white-box cryptographic models with $S^{(1)} AS^{(2)}$ structures. 

\section{WHITE-BOX CRYPTOGRAPY AND STRUCTURAL ANALYSIS}
White-box cryptographic model uses obfuscation techniques by applying encoding to plaintext, ciphertext, and intermediate values. Various white-box cryptographic model design studies are underway, starting with the symmetric key cryptography AES white-box cryptographic model proposed in 2002 [2]. This white-box cryptographic model is very closely related to cryptographic logic, structural analysis. Structural analysis is a study that started with a different motive than white-box cryptography. 
This section examines the research trend of the white-box cryptographic model. In addition, we look at the structural analysis research trend closely related to the security of white-box cryptography. 
 
 \subsection{RESEARCH TRENDS IN WHITE-BOX CRYPTOGRAPHY}
 Chow et al. proposed AES’ white-box cryptographic model in 2002 and presented a design idea that binds fixed encryption keys into tables, including XOR (exclusive-or) operations. Chow's design ideas are the basis for designing the white-box cryptographic model to date. However, it is not safe with a BGE attack [9], and encryption keys can be extracted by analyzing the table reference method regardless of whether encoding and obfuscation are applied. This is enough to obtain an encryption key from a given table in a few seconds in a PC environment. 
After that, there have been various studies to supplement Chow's white-box design, but most attack methods have been proposed within a few years. Xiao, Lai [10] presented 16-bit, 32-bit linear function encoding to improve the weakness of 4-bit unit non-linear encoding in the table reference method. Still, a linear equivalence transformation attack method was discovered by Mulder et al. [11]. In addition, in 2020, vulnerabilities were found in the method of obfuscating the round boundary and adding dummy rounds proposed by Xu et al. [12]. As shown in TABLE I, research on white-box cryptography, which has been steadily improved in table reference methods, has continued until recently [13]. 

% Insert table  1 here
\begin{table}
\caption{Designs and Attacks in Whitebox cryptography}
\label{table}
\setlength{\tabcolsep}{3pt}
\begin{tabular}{|p{120pt}|p{50pt}|p{50pt}|}
\hline
Whitebox Cryptography& 
Design& 
Attack \\
\hline
Whitebox AES & Chow (2002) & Billet (2004) \\
Whitebox DES & Chow (2002) & (2007) \\
Perturbated White-box AES & (2006) & (2010) \\
White-box AES with large linear encoding &
(2009) & (2013) \\
\hline
\end{tabular}
\label{tab1}
\end{table}

To overcome the limitations of white-box cryptography for standard cryptography, research is also underway to propose white-box cryptography and a suitable cryptographic algorithm. This started in earnest with introducing the space-hard concept by Bogdanov et al. [14] in 2015. WEM (standing for white-box Even-Mansuor) of Chow et al. [26] proposed a new security concept and operation mode of white-box cryptography. Kwon et al. [27] announced FPL (Feistel cipher using Parallel table Look-ups) block ciphers that combine provable security using parallel table reference methods. Along with developing algorithms suitable for this white-box, the security concept was also discussed from various perspectives. Wyseur [4], Saxena [28] in 2009, and Delerablee et al. [29] in 2013 summarized the security concept that white-box cryptography should satisfy, but most of them are difficult to achieve. In 2020, Bock et al. [30, 31] proposed a security concept considering a practical environment and summarized the security of white-box cryptography based on HW-binding and SW-binding. There are various viewpoints on the security concept and goal of white-box cryptography, and commercial products mainly use private white-box cryptography technology that combines solid obfuscation [5, 32, 33]. 
The white-box cryptography design is also utilized in SM4, a Chinese standard block cipher algorithm. Various designs and analyses of white-box cryptography are in progress in China. Xioa, Lai [18] and Shang [24] and Yao, Chen [25] designed a white-box cryptography model based on SM4 in 2009, 2016, and 2020, but based on a collision attack, the results were announced that it is difficult to maintain security through the analysis method [19]. As a similar research case, an SM4-based light-weight white-box cryptography model suitable for WSNs (Wireless Sensor Networks) environment was proposed by Shi, Yang et al. in 2015. In 2019, a light-weight white-box cryptographic model suitable for distributed resource systems and combining non-linear and affine functions was proposed [22, 8]. However, a vulnerability in the white-box cryptographic model was discovered in WSNs through collision-based attacks in 2021 [23]. The white-box cryptographic model proposed in 2019 can confirm its applicability to structural analysis attack [7]. 

\subsection{RESEARCH TRENDS IN STRUCTURAL ANALYSIS}
White-box cryptography is closely related to cryptographic logic, structural analysis. In 1997, Paratin et al. [34] attempted to create the function of public-key cryptography by combining S-box, which is the secret key cryptography logic, and higher-order polynomials. However, although it did not yield successful results, it led to a systematic structural analysis study in the future. As shown in TABLE II, security analysis is conducted on functions of various structures in which nonlinear and affine layers (or linear) with multiple S-boxes alternately appear. 

% Insert table  2 here
\begin{table}
\caption{Structural analysis of Substitution-Affine Iterations}
\label{table}
\setlength{\tabcolsep}{3pt}
%\begin{tabular}{|p{25pt}|p{75pt}|p{115pt}|}
\begin{tabular}{|p{20pt}|p{140pt}|p{60pt}|}
\hline
Year& 
Topic& 
Authors \\
\hline
2001 
& Structural cryptanalysis of SASAS 
& A Biryukov et al. \\
2003 
& Affine Equivalance Algorithm 
& A Biryukov et al. \\
2015
& Structural cryptanalysis of ASASA 
& I Dinur et al. \\
2015
& Analytic Tools for White-box Cryptography 
& C.H. Baek et al. \\
2018 
& An improved Affine Equivalence Algorithm
& I Dinur \\
\hline
\end{tabular}
\label{tab1}
\end{table}

Structural analysis is a study of a method of determining each component under conditions in which the structure of a function is known, but the specific function of each component is unknown. In other words, it is a technique of creating an equivalent function having the same function using only the input/output value of a given oracle function. In 2001, Biryukov et al. [6] considered a function of the SASAS structure as an oracle and discovered a way to find an oracle of the same structure with equal functionality. Using this method, you can discover the encryption key hidden inside the SASAS structure. Most of the white-box cryptography using the table reference method can be attacked by this analysis method. The BGE attack [9] can also be interpreted as this analysis method. Baek et al. proposed a toolbox that generalized structural analysis and presented systematic and quantitative attacks on various structures. This structural analysis has expanded its research to various structures such as ASASA and SASASASAS. 
Typical attacks on white-box cryptography include obtaining encryption keys and attacking one-wayness properties by constructing a decryption algorithm for a given encryption system. White-box cryptography, which combines non-linear and affine functions into tables, is difficult to maintain security through structural analysis. However, various studies are still in progress to design a white-box encryption model based on one-wayness. 

\section{SHI'S WHITE-BOX Encryption Scheme: LW-WBES}
In 2019, Shi et al. proposed a white-box encryption scheme
for light-weight embedded devices including mobile phones and navigating systems. 
We denote their scheme by LW-WBES, which means a Light-Weight White-Box Encryption Scheme.
LW-WBES has the following features:
\begin{itemize}
\item The block size (input/output size) is 120 bits. 
\item The number of rounds depends on the security level such as
16(default), 10(aggressive), or 32(conservative).
\item The encryption process is designed as a variant of Feistel network.
\item Two types of keys (black-box key and white-box key) are used for
providing black-box security and white-box security simutaneously. 
Hence, the key size of LW-WBES is extremely large.
\end{itemize}

\subsection{Design rationale}
In order to overcome the difficulties of white-box implementations
of standard ciphers, Shi et al. propose a new white-box friendly cipher secure 
against white-box attack context.
Their design stretergies can be summarized as follows:
\begin{itemize}
\item The scheme has the secret components
based on the Feistel network, which protect the secret white-box keys from
white-box attacks including DCA and DFA.
\item Since components of three different size (4, 5, 6-bit) are integrated,
it is hard to mount the structural analysis directly.
\item The secret components can be reused in each round for saving memory usage
in light-weight devices.
\item The scheme does not require additionaly external encodings
nor obfuscation techniques. 
\end{itemize}
We will show that the goal of design rationale cannot be satisfied
and weak against structural cryptanalysis in particular.

\subsection{Specification}

LW-WBES is a 120-bit block cipher and the number of round
can be chosen based on the level of security and the constraints of resources.
Here, we describe the encryption process of 16-round default version.
120-bit plaintext $PT=(L,R)$ is input  for the Feistel network divided into
5-bit variables as:
\[
PT = (L,R) = (L_0, L_1, \ldots, L_{11}, R_0, R_1, \ldots, R_{11}),
\]
where 
$L_i, R_i \in GF(2)^5 \text{ for } i=0,1, \ldots, 11$.
In each round, the round function $F: GF(2)^{60}\times GF(2)^{72} \to GF(2)^{72}$
consumes 72-bit black-box round key $rk$ by
\[
F: (x,k) \mapsto (\Theta_0(x)\oplus rk_0, \ldots, \Theta_{11}(x)\oplus rk_{11}),
\]
where $\Theta_i$ are nonlinear surjective functions whose outputs are 8-bits 
and $rk$ is divided into 6-bit components $rk_i$ for $i=0,1,\ldots, 11$.
In fact, we do not need the details of $\Theta_i$ to construct our attack algorithm.
Instead of usual mixing by exclusive-or in Feistel network,
LW-WBES uses nonlinear mixing function called T-box which contains
white-box key component.
The round transformation $(X_L, X_R) \mapsto (Y_L, Y_R)$ can be written as
\[
\begin{cases}
Y_L = X_R, \\
Y_R = T(X_L, F(X_R,rk)),
\end{cases}
\]
where T-box $T: GF(2)^{60} \times GF(2)^{72} \to GF(2)^{60}$  consists of
4 sub-components $G$, $F'$, $H^*$, and $M$ is defined as
\begin{equation}\label{eq-T-box}
T(x,y) = H^*(M(G(x)\oplus F'(y)))
\end{equation}

\begin{itemize}
\item $G: GF(2)^{60} \to GF(2)^{60}$ is composed of 12 bjiections
$G_0, G_1, \ldots, G_{11}$ in parallel so that
\[
(x_0, x_1, \ldots, x_{11}) \stackrel{G}{\mapsto} (G_0(x_0), G_1(x_1),\ldots, G_{11}(x_{11})).
\]
\item $F': GF(2)^{72} \to GF(2)^{60}$ takes output of round function $F$ and
squizes them into 60-bits.
\[
(y_0, y_1, \ldots, y_{11}) \stackrel{F'}{\mapsto} (F'_0(y_0), F'_1(y_1),\ldots, F'_{11}(y_{11})).
\]
\item $M: GF(2)^{60} \to GF(2)^{60}$ is an invertible linear transformation represented
by a binary matrix $M$ whose 5-bit columns are $M_0, M_1, \ldots, M_{11}$.
\[
M = \left( M_0 M_1 \cdots M_{11} \right), 
\]
where $M_j$ are $60\times 5$ submatrices for $j=0,1,\ldots, 11$.
\item $H^*: GF(2)^{60} \to GF(2)^{60}$ is composed of fifteen 4-bit bijections
$(h^*_0, h^*_1, \ldots, h^*_{14})$.
\end{itemize}
The component functions $F'$, $G$, $M$, and $H^*$ are
white-box keys that cannot be exposed during the white-box implementation of encryption process.

In the white-box encryption algorithm,
evaluations of T-box $T$ are possible without knowing its component functions (white-box keys),
since $T$ is implemented as several steps of table look-ups.
In fact, T-box $T$ without final transformation $H^*$, 
say $\tilde T(x,y):= M(G(x)\oplus F'(y))$, can be represented
by the tables of 12-bit input and 60-bit output as follows:
\[
\tilde T: GF(2)^{60} \times GF(2)^{72} \to GF(2)^{60}, 
\]
where its first input is $x = (x_0, x_1, \ldots, x_{11})$  
and the second is $y = (y_0, y_1, \ldots , y_{11})$.
Then we can rewrite $\tilde T$ as  
\begin{gather*}
\tilde T: GF(2)^{11} \to GF(2)^{60},\\
\left( (x_0, y_0), (x_1, y_1), \ldots, (x_{11}, y_{11}) \right)
\stackrel{\tilde T}{\mapsto} (z_0, z_1, \ldots, z_{14}).
\end{gather*}
Define $\tilde T_j(x_j, y_j) = M_j( G(x_j) \oplus F'(y_j))$ for
$0\le j \le 11$.
Then 
\begin{align*}
\tilde T(x,y) & := M(G(x)\oplus F'(y)) \\
&= M_0(G_0(x_0)\oplus F'_0(y_0)) \oplus \cdots \\
& \quad\quad \cdots\oplus M_{11}(G_{11}(x_{11}) \oplus F'_{11}(y_{11})) \\
&= \tilde T_0(x_0, y_0) \oplus \cdots \oplus \tilde T_{11}(x_{11}, y_{11}).
\end{align*}
Note that each $\tilde T_j(x_j, y_j)$ can be implemented
as a pre-computed table with $2^{11}$ entries which takes 15,360 bytes.
On the other hand, 60-bit output of $\tilde T$ can be divided into 
fifteen 4-bit subblocks $(z_0, z_1, \ldots, z_{14})$.
Apply a 4-bit random nonlinear bijection $h_k$ on each $z_k$ ($k=0,1,\ldots, 14)$
in the output of lookup table $\tilde T_j(x_j, y_j)$. Then
we have to use a cascade structure of masked adders 
to obtain the final output of $T$ including $H^*$ layer, 
as depicted in Chow's WB-AES.

The ciphertext of LW-WBES is produced by iterating this process 16 times
with their corresponding black-box and white-box keys at each round.
Note that there are no output encoding at the end of the final round,
since nonlinear bijections appear in the T-box $T$.

\subsection{White-box Implementation and its Security}

In order to hide white-box keys in T-box $T$,
LW-WBES uses 12 tables $T_i: GF(2)^5 \times GF(2)^6 \to GF(2)^{60}$
for $i=0,1,\ldots, 11$ definded as
\[
T_i(x_i, y_i) := H'_i(\tilde T_i(x_i, y_i)),
\]
where $H'_i$ consists of 4-bit random bijections
$(h'_{i,0}, \ldots, h'_{i,14})$ so that
each output can be written as
\[
T_i(x_i, y_i) = (h'_{i,0}(z_{i,0}), h'_{i,1}(z_{i,1}), \ldots, h'_{i,14}(z_{i,14})).
\]
In fact, T-box produces output end with nonlinear bijection $H^*$ as \eqref{eq-T-box}:
\begin{align}
T(x,y) &= H^*(M(G(x)\oplus F'(y))) \nonumber \\
&= H^*(\tilde T(x,y)) \nonumber \\
&= H^*\left( \tilde T_0(x_0, y_0) \oplus \cdots \oplus \tilde T_{11}(x_{11}, y_{11})
\right) \nonumber \\
&= H^*\Big( H'^{-1}_0\circ T_0(x_0, y_0) \oplus \cdots \nonumber \\
& \quad\quad\quad \cdots \oplus H'^{-1}_{11}\circ T_{11}(x_{11}, y_{11})
\Big)   \label{eq-2}
\end{align}
Note that exclusive-or operations can be considered as 
operations on each 4-bit components. 
For instance, the first 4-bit output of $T(x,y)_{[0:3]}$ is
computed as
\[
h^*_0\left(
h'^{-1}_{0,0}(T_0(x_0,y_0)_{[0:3]}) \oplus \cdots
\oplus h'^{-1}_{11,0}(T_{11}(x_{11},y_{11})_{[0:3]})
\right).
\]
When we have two masked variables $w_1=h_1(z_1)$ and $w_2=h_2(z_2)$
with random bijection $h_1$ and $h_2$, respectively,
the masked exclusive-or with bijection $h^*$ can be computed by
\begin{align}
h^*(z_1 \oplus z_2)
& = h^*\left(
h_1^{-1}(h_1(z_1)) \oplus h_2^{-1}(h_2(z_2)) \right) \nonumber \\
&= h^*\left( h_1^{-1}(w_1) \oplus h_2^{-1}(w_2) \right) \label{eq-3}
\end{align}

If each variable in \eqref{eq-3} represents $n$-bit data,
$2n$-bit to $n$-bit lookup table $L_{h^*}: (w_1, w_2) \mapsto h^*(z_1 \oplus z_2)$
hide its internal components $h_1$, $h_2$, and $h^*$.
This technique called `Masked Adder' enables us to obtain $T(x,y)$
without revealing $H^*$ by using a cascade of lookup tables.

In the white-box implementation of T-box,
the encrytopn process accesses the lookup tables
$T_0, \ldots, T_{11}$ and masked adders.
Hence, encryptor (as well as white-box adversary) cannot extract
the white-box keys such as $G$, $F'$ and $H^*$.
Furthermore, LW-WBES is designed to provide one-wayness.
That is, it is not feasible to perform decryption process 
only with the white-box implementation of encyption.

To sum up, LW-WBES is claimed to be secure 
against black-box attack as well as white-box attack contexts.
Its Feistel structure with black-box round keys  resists against
black-box attack and a large set of white-box keys implemented in T-boxes
enhances the security against white-box adversaries.
However, we will show later that it is possible to recover the plaintext from 
the corresponding ciphertext by accessing white-box encryptin orable only.
In fact, inverting T-box can be done by structural analysis of the SAS variant
explained in the next section.

\section{Structural analysis}

For a given black-box function $F$,
the goal of structrural analysis is to reveal its internal components explicitly.
informally,
\begin{quote}
Given a function $F$ with known internal structure,
{\bf structural analysis} is defined as the analysis of $F$
to find an equivalent function $F^*$ by determining its internal components explicitly.
\end{quote}

Suppose that we have a bijective function $F$ as
\[
f:GF(2)^N \to GF(2)^N.
\] 
%where $N = m\times k$.
Additionally, we assume following conditions on $F$:
\begin{itemize}
\item Given $x$, anyone can compute $F(x)$ with ease.
\item Positive integer $N$ is large enough so that it is infeasible to invert $F$.
i.e., for a given $y$, it is not possible to find $x$ satisfying 
$F(x) = y$ within a reasonable time. 
\item The number of its internal components and the size of input and output
of each component are known to public. 
\item The structure(how to combine components) of $F$ is open to public
but each component itself is not known except for its input and output sizes.
Thue, given $x$, we merely calculate $F(x)$ without knowing intermediate values.
\end{itemize}

Last two decades, 
the structrural analysis has been studied for functions with layered structure.
In 2002, Biryukov et al. proved that a function of SASAS structure
can be analyzed successfully so that one can construct an equivalent function 
explicitly, where $S$ layer is composed of small nonlinear S-boxes in parallel
and $A$ layer is an affine or a linear transformation.
Later, several types of structural analysis such as ASASA have been considered.


For a given $F$, our goal for structural analysis is 
to find an equivalent function $F^*$ explicitly.

\subsection{Multiset properties}

We define a multiset by a set 


\subsection{Structural analysis of $S^{(1)}AS^{(2)}$}

Multiset properties play a key role in the structural analysis.
For example, suppose that $N$-bit function $F$ has SASAS structure
and each $S$-layer consists of $m$-bit S-boxes.
Then if we use input multiset of the form $(P,C,C,\ldots, C)$,
output  of $F$ has B(balanced) property [Biryukov].

We focus on the $S^{(1)}AS^{(2)}$ structure for T-box function.
LW-WBES uses a variant of SAS structure for T-box which
consists of two $S$ layers with different size of S-boxes and a linear function
between them.
In this section, we focus on bijective functions with $S^{(1)}AS^{(2)}$ structure as follows:
\begin{itemize}
\item $S^{(1)}: GF(2)^{m_1 \cdot k_1} \to GF(2)^{m_1 \cdot k_1}$ 
\[
S^{(1)}(x_0, \ldots, x_{k_1-1}) = s^{(1)}_0(x_0) \| \cdots \| s^{(1)}_{k_1-1}(x_{k_1-1}).
\]
\item $A: GF(2)^{m_1 \cdot k_1} \to GF(2)^{m_1 \cdot k_1}$  is a linear transformation
on the vector space over $G(2)$ which can be represented by
an $m_1 k_1 \times m_1 k_1$ matrix.
\item $S^{(2)}: GF(2)^{m_2 \cdot k_2} \to GF(2)^{m_2 \cdot k_2}$ 
\[
S^{(2)}(x_0, \ldots, x_{k_2-1}) = s^{(2)}_0(x_0) \| \cdots \| s^{(2)}_{k_2-1}(x_{k_2-1}).
\]
\end{itemize}
Since the function is a bijection, we observe that
\[
N:=m_1\cdot k_1 = m_2\cdot k_2.
\]
LW-WBES chooses $m_1 = 5$, $k_1 =12$ and
$m_2=4$, $k_2 = 15$ so that $N=5\times 12 =4\times 15= 60$ bits.

We can remove $S^{(2)}$-layer efficiently by Theorem \ref{thm-4-1}.
\begin{theorem}\label{thm-4-1}
For a given function $F := S^{(2)}\circ A\circ S^{(1)}$,
we can find a function $\tilde{S}^{(2)}$ such that
\[
({\tilde{S}^{(2)}})^{-1} \circ F = \tilde A \circ S^{(1)}.
\]
\end{theorem}
{\it proof. \ } 
Choose $2^{m_1}$ input data %$\{X_i : i=0,1,\ldots, 2^{m_1}-1 \}$
\[
X_i := (x_{0,i}, x_{1,i}, \ldots, x_{k_1-1,i}), \quad i=0,1,\ldots, 2^{m_1}-1
\]
that form a multiset with property
$(D, D, \ldots, D)$.
Then $S^{(1)}$-layer preserves the multiset property and
the input of $S^{(2)}$-layer has balanced property $(B,B,\ldots, B)$.
The output of $F$ can be written as
\[
Y_i:=(y_{0,i}, y_{1,i}, \ldots, y_{k_2-1,i}), \quad i=0,1,\ldots, 2^{m_1}-1.
\]
It follows from the above balanced property that
\begin{equation}\label{eq-4}
\bigoplus_{i=0}^{2^{m_1}-1} (s_k^{(2)})^{-1}(y_{k,i}) = 0, \quad
k=0,1,\ldots, k_2-1.
\end{equation}
In order to determine the component $s_k^{(2)}$, we introduce new variables
$z_{k,0}, z_{k,1}, \ldots, z_{k,2^{m_2}-1}$ such that
\[
z_{k,j} = (s_k^{(2)})^{-1}(j), \quad j=0,1,\ldots, 2^{m_2}-1.
\]
For a fixed $k$, the equation \eqref{eq-4} can be interpreted as
the equation for unknowns $z_{k,0}, z_{k,1}, \ldots, z_{k,2^{m_2}-1}$.
Choose another input multiset and repeating this process 
until we obtain ``{\it sufficiently many}'' linearly independent equations.
Then we can pick a solution of the system of linear equations,
which means that we determine the internal component $s_k^{(2)}$ of $F$.
Note that the solution is not unique. If we find a solution $s_k^{(2)}$,
then  $s_k^{(2)}\circ a_k$ is also a solution, where $a_k$ is an affine map.


On the other hand, suppose that $F$ is designed as
\[
F= S^{(2)}\circ A\circ S^{(1)}
\]
and we also know all functions used as internal components.
Then it is easy to find a class of equivalent functions
\begin{equation}\label{eq-5}
F^* = \tilde S^{(2)} \circ \tilde A \circ S^{(1)},
\end{equation}
by inserting an affine layer $a^*$ and its inverse between $S^{(2)}$ and $A$,
where
\[
a^* = (a_0^*, a_1^*, \ldots, a_{k_2-1}^*),
\]
and $a_ k^*$ ($k=0,1,\ldots, k_2-1$) is an $m_2$-bit affine bijection.
If we choose $\tilde S^{(2)} := S^{(2)} \circ (a^*)^{-1}$ and $\tilde A := a^*\circ A$,
then $F^* = F$ as a black-box function.
Thus each $s_k^{(2)}$ has ${\cal N}(m_2)$ equivalent forms,
where ${\cal N}(m_2)$ is the number of $m_2$-bit affine bijections.

Reconsider the meaning of ``{\it sufficiently many}'' mentioned  above in the proof.
If we collect $2^{m_2}-m_2-1$ linearly independent equations for $z_{k,j}$'s,
then a system of linear equations of the form \eqref{eq-4}
has $m_2+1$ dimensional kernel by the rank-nullity theorem [Strang].
Then there are $m_2+1$ free variables among $z_{k,j}$'s.
We have ${\cal N}(m_2)$ possible cases since we have to choose
them so that $s_k^{(2)}$ is invertible.

To sum up, if we have $2^{m_2}-m_2-1$ linearly independent equations for for $z_{k,j}$'s
and choose a solution to define $\tilde s_k^{(2)}$, then
there exists an affine bijection $a_k^*$ such that
\[
\tilde s_k^ {(2)} = s_k^{(2)} \circ a_k^*.
\]
Set 
\begin{align*}
\tilde S^{(2)} & := \tilde s_0^{(2)} \| \cdots \| \tilde s_{k_2-1}^{(2)}, \\
\tilde A &:= \left((a_0^*)^{-1} \| \cdots \| (a_{k_2-1}^*)^{-1}\right)\circ A.
\end{align*}
Then
$ F = {\tilde{S}^{(2)}} \circ \tilde A \circ S^{(1)}$.
This completes the proof. \hfill $\square$

Note that we can construct input multisets $(D,D,\ldots, D)$ easily
by adding two layers at the begining of $F$ so that
the resulting function has $SASAS$ structure.
Applying input as $(P, C, C,\ldots, C)$, we expect the same result.
Thus we can make as many input multisets as we wish 
by choosing constant parts arbitrarily.
Thus it is easy to collect $2^{m_2}-m_2-1$ lineary independent equations.
Algorithm \ref{alg-1} provides a way to make $S^{(2)}\circ A \circ S^{(1)}$ to
$\tilde A \circ S^{(1)}$.

\begin{breakablealgorithm}
    \setstretch{1.2}
    \caption{Recovering $S^{(2)}$-layer}
    \label{alg-1}
    \textbf{Input:} $F = S^{(2)}\circ A \circ S^{(1)}$ as a black-box function \\
    \textbf{Output:} $\tilde S^{(2)}$ such that 
    $(\tilde S^{(2)})^{-1} \circ F = \tilde A\circ S^{(1)}$
    \begin{algorithmic}[1]
    \For  {$k=0$ to $k_2-1$}
    \State Assign variables $z_j :=(s_k^{(2)})^{-1}(j)$ for $0\le j<k_2$.
    \State $A \leftarrow \phi$ \Comment{$A$: Set of equations}
    \While {$n < 2^{m_2} - m_2 -1$}
    \State Choose a multiset with property $(D,\ldots, D)$ as
     \begin{flushright}\footnotesize
	$\{ X_i:=(x_{0,i}, \ldots, x_{k_1-1,i}) :  i=0,1,\ldots, 2^{m_1}-1 \}.$
	\end{flushright}
    \State Store the corresponding output as $Y_i = F(X_i)$
     \begin{flushright}\footnotesize
	$\{ Y_i:=(y_{0,i}, \ldots, y_{k_1-1,i}) :  i=0,1,\ldots, 2^{m_1}-1 \}.$
	\end{flushright}
    \State Construct an equation $(eq)$ as
     $$(eq): \ \bigoplus_{i=0}^{2^{m_1}-1} (s_k^{(2)})^{-1}(y_{k,i}) = 0$$
    \State \If {$(eq)$ is linearly independent of $A$} 
		\State {Add the equation $(eq)$ to $A$}
    \EndIf
    \EndWhile
    \State Solve the system of linear equations in $A$:
    \begin{center} 
      Determine $z_j$ by Gaussian elimination.
    \end{center}
    \State Define $\tilde s_k^{(2)}(z_j) = j$.
    \EndFor
    \State \Return $\tilde S^{(2)} \leftarrow \tilde s_0^{(2)} \| \cdots \| \tilde s_{k_2-1}^{(2)}$. 
    \end{algorithmic}
\end{breakablealgorithm}

From Algorithm 1, we can successfully remove the $S^{(2)}$-layer and
obtain $\tilde F$ with SA-structure.
Considering the differential charateristic of $\tilde F$, we can remove its
linear part, too.
In the following two lemmas, we recall elementary concepts of linear algebra such as
rank of matrices and difference preserving property of linear transformation.

\begin{lemma}\label{lem-4-1}
Let $L_A:GF(2)^m\to GF(2)^n$ and $L_B:GF(2)^n\to GF(2)^n$ be linear transformations
whose corresponding matrices are $A$ and $B$, respetively.
Then we have the associated property of the rank 
\[
 \rank(L_B\circ L_A) \le \rank(L_A).
\]
\end{lemma}

\begin{lemma}\label{lem-4-2}
Let $L:GF(2)^n\to GF(2)^n$ be a linear transormation.
Then $L$ preserve the difference as $\Delta L(x) = L(\Delta x)$.
More precisely, for input data $x_1$, $x_2$ and their difference $\Delta x := x_1\oplus x_2$,
\[
\Delta L(x) = L(x_1) \oplus L(x_2) = 
L(x_1\oplus x_2) = L(\Delta x).
\]
\end{lemma}

\begin{theorem}\label{thm-4-2}
For a given $\tilde F = \tilde A \circ S^{(1)}$,
we can find a function $\tilde{A^*}$ such that
\[
(\tilde A^*)^{-1} \circ \tilde F = \tilde S^{(1)}.
\]
\end{theorem}

{\it proof.} \
Let $U_{in}$ be a set of all possible differences
for $GF(2)^{m_1}$.
Initialize two sets $\Delta_{in}$ and $\Delta_{out}$ as empty, which 
will store input and output differences.
Randomly select $d_{in}$ from $U_{in}$ and
construct a pair of input for $\tilde F$ with its difference $D_{in}:=(d_{in},0,\ldots, 0)$.
Then we have the corresponding output difference $\Delta Z:=\tilde F(D_{in})$  as
\[
D_{out} := \tilde F(x_0,0,\ldots, 0) \oplus \tilde F(x_1,0,\ldots, 0).
\]
Update two setes as
\[
\Delta_{in} \leftarrow \Delta_{in} \cup \{D_{in}\}, \quad
\Delta_{out} \leftarrow \Delta_{out} \cup \{D_{out}\}.
\]

Repeat this process from selecting new $d_{in}$ and
update $\Delta_{in}$ and $\Delta_{out}$ only if
$D_{out}$ is linearly independent of differences which are already stored in $\Delta_{out}$.
We can check this by comparing the ranks of the matrices generated by $\Delta_{out}$
and $\Delta_{out} \cup \{ D_{out} \}$.
Suppose that we have $\ell$ output differences in $\Delta_{out}$ as
\[
\Delta_{out} = \{ w_0, w_1, \ldots, w_{\ell-1} \},
\]
where $w_i$ are interpreted as column vectors for $i=0,1,\ldots, \ell-1$
and we have a new difference $w_{\ell}:=\tilde F(D_{in})$ to be added in $\Delta_{out}$.
If the rank increases, 
\[
\rank\left[  w_0 \cdots w_{\ell-1} w_{\ell} \right]
= \rank\left[  w_0 \cdots w_{\ell-1} \right] + 1,
\]
then update $\Delta_{out} \leftarrow \Delta_{out} \cup \{w_{\ell}\}$.

Iterating this process,
we can reach the maximal rank $m_1$ and
obtain $m_1$ output differences in 
$\Delta_{out} = \{ w_0, w_1, \ldots, w_{m_1-1} \}$.

Note that by Lemma, \ref{lem-4-1} the maximal rank is $m_1$,
since we make input differences in $m_1$-bit S-box $s_0^{(1)}$.


Consider two inputs $(x_1,0,\ldots, 0)$ and $(x_2,0,\ldots, 0)$ with
$d_{in} := x_1 \oplus x_2$.
Since $\tilde F$ is a black-box function,
we do not know the intermediate difference 
\[
\delta y := S^{(1)}(D_{in}) = S^{(1)}(x_1,0,\ldots,0) \oplus S^{(1)}(x_2,0,\ldots,0).
\]
However, we have the final difference $D_{out}$ interpreted as
\[
D_{out} = \tilde F(D_{in}) = A(\delta y).
\]
With $\Delta_{out} = \{ w_0, w_1, \ldots, w_{m_1-1} \}$,
let $\delta y_0, \delta y_1, \ldots, \delta y_{m_1-1}$ be the corresponding
intermediate differences.
Then the linear transformation $\tilde A$ maps
\[
(\delta y_j, 0,\ldots, 0) \mapsto w_j,
\quad\text{for } j=0,1,\ldots, m_1-1.
\]
There is a linear map $L_0 :GF(2)^{m_1} \to GF(2)^{m_1}$ such that
\[
\delta y_j \mapsto \hat e_j,
\]
where $\hat e_j = (e_{j0}, e_{j1}, \ldots, e_{jm_1-1})$ defind as
$e_{ji} = 1$ for $i=j$ and $e_{ji} = 0$, otherwise.

Define a linear map $A^*_0: GF(2)^{m_1} \to GF(2)^n$ by
\[
A^*_0(x) := \tilde A(L_0(x), 0,\ldots, 0).
\]
Its matrix representation is $W_0:=\left[  w_0 \cdots w_{m_1-1} \right]$.

In a similar way, we successively define linear maps $A^*_j$ 
and matrices $W_j$ for $j=1,\ldots, k_1-1$.
Combining these matrices, we can define $\tilde A^*:GF(2)^n \to GF(2)^n$ as
\[
\tilde A^*(x_0, \ldots, x_{k_1-1}) := \tilde A(L_0(x_0), \ldots, L_{k_1-1}(x_{k_1-1}))
\]
represented by $n\times n$ matrix $\left[W_0 \cdots W_{k_1-1} \right]$.

It is easy to check $(\tilde A^*)^{-1}\circ \tilde F$ is 
divided into  $k_1$ S-boxes in parallel.
In fact, since $\tilde A$ is linear, we may write
\begin{align*}
(\tilde A^*)^{-1}\circ \tilde F
&= (L_0^{-1}\| \cdots \| L_{k_1-1}^{-1}) \circ \tilde A^{-1}
\circ \tilde A \circ S^{(1)}  \\
&=  (L_0^{-1}\| \cdots \| L_{k_1-1}^{-1}) \circ S^{(1)}  \\
&= (L_0^{-1}\circ s_0^{(1)} \| \cdots \| L_{k_1-1}^{-1}\circ s_{k_1-1}^{(1)}) \\
&= (\tilde s_0^{(1)} \| \cdots \| \tilde s_{k_1-1}^{(1)}). 
\end{align*}
Without knowing the functions $L_j$'s in the middle, we
can remove the linear part $\tilde A$.
\hfill $\square$.  

Applying Theorem \ref{thm-4-1} and \ref{thm-4-2}, 
we can find all internal components of the black-box function $F$ 
accodring to the steps in Algorithm \ref{alg-2}.
Note that all components of the function $F$ cannot be determined uniquely.
So far, we have shown that SAS-structure can be successfully analyzed even though
S-boxes of different size are used in S-layers.
%== fig? Since there are many equivalent form

\begin{breakablealgorithm}
    \setstretch{1.2}
    \caption{Recovering $A$-layer}
    \label{alg-2}
    \textbf{Input:} $\tilde F := \tilde A \circ S^{(1)}$ as a black-box function \\
    \textbf{Output:} $\tilde A^*$ such that 
    $(\tilde A^*)^{-1} \circ \tilde F =\tilde S^{(1)}$
    \begin{algorithmic}[1]  
   \State Let $U_{in}$ be a set of all possible input differences for an $m_1$-bit function.
    \For  {$k=0$ to $k_1-1$}
    \State $\Delta_{in} \leftarrow \phi$
    \State $\Delta_{out} \leftarrow \phi$
    \Repeat
    \State Select $d_{in} \leftarrow U_{in}$ randomly.
    \State $D_{in}    \leftarrow (0,\ldots, d_{in}, \ldots, 0)$
   \If {$\rank\tilde F(\Delta_{in}) < \rank\tilde F(\Delta_{in}\cup \{D_{in}\})$}
    \State $\Delta_{in} \leftarrow \Delta_{in} \cup \{ D_{in} \}$
    \State $\Delta_{out} \leftarrow \tilde F(\Delta_{in})$
   \EndIf
\Until {$\rank(\Delta_{out}) = m_1$}
   \State From the set $\Delta_{out} = \{ w_0, w_1, \ldots, w_{m_1-1}\}$, define
   \[
       W_k := \left[ w_0 \cdots w_{m_1-1} \right].
   \]
   \EndFor
   \State Define a linear map $\tilde A^*$ using $n\times n$ matrix $W$:
   \[
        W:= \left[ W_0 \cdots W_{k_1-1} \right].
   \]
   \State Inverting the matrix $W$, define $\tilde S^{(1)}$ as
    \[
    \tilde S^{(1)}:= (\tilde A^*)^{-1}\circ \tilde F = (\tilde s_0^{(1)} \| \cdots \| \tilde s_{k_1-1}^{(1)}).
    \]
    \State \Return $\tilde S^{(1)}$.
    \end{algorithmic}
\end{breakablealgorithm}


\section{Cryptanalysis of Shi's Algorithm}


From the result of analysis in Section 4, we construct an attack algorithm for
Shi's LW-WBES.
The type of our attack is ``plaintext recovery attack'':
\begin{quotation}
For a given implementation of LW-WBES and ciphertexts,
we can find its plaintexts by inverting each round successively.
\end{quotation}





\begin{itemize}
\item round inversion
\item Attack algorithm of plaintext recovery attack
\item Experimental results
\item Possible countermeasures
\end{itemize}


\section{Conclusion}
A conclusion section is not required. Although a conclusion may review the 
main points of the paper, do not replicate the abstract as the conclusion. A 
conclusion might elaborate on the importance of the work or suggest 
applications and extensions. 

\section*{Acknowledgment}

The preferred spelling of the word ``acknowledgment'' in American English is 
without an ``e'' after the ``g.'' Use the singular heading even if you have 
many acknowledgments. Avoid expressions such as ``One of us (S.B.A.) would 
like to thank $\ldots$ .'' Instead, write ``F. A. Author thanks $\ldots$ .'' In most 
cases, sponsor and financial support acknowledgments are placed in the 
unnumbered footnote on the first page, not here.



\begin{thebibliography}{00}

%=====
\bibitem{chow_wbaes} 
S. Chow, P. A. Eisen, H. Johnson, and P. C. van Oorschot, 
``White-box cryptography and an AES implementation'', 
\emph{SAC 2002}, LNCS volume 2595, 2003. 

\bibitem{chow_wbdes} 
S. Chow, P. A. Eisen, H. Johnson, and P. C. van Oorschot, 
``A white-box DES implementation for DRM applications'', 
\emph{Security and Privacy in Digital Rights Management, ACM CCS-9 Workshop, DRM 2002}, LNCS volume 2696, 2003.

\bibitem{wysuer2009}
B. Wyseur, 
``White-Box Cryptography'', 
PhD thesis, Katholieke University Leuven, 2009.

\bibitem{biryukov_sasas}
A. Biryukov, A. Shamir, 
``Structural cryptanalysis of SASAS'', 
\emph{Eurocrypt 2001, J. of Cryptology}, 23(4), 2010.

\bibitem{yim2021}
H. Yim, J.-S. Kang, Y. Yeom, 
``An Efficient Structural Analysis of SAS and its Application to White-Box Cryptography'', 
\emph{IEEE TENSYMP}, 2021.

\bibitem{shi2019}
Y. Shi, W. Wei, H. Fan, M. H. Au and X. Luo, 
``A Light-Weight White-Box Encryption Scheme for Securing Distributed Embedded Devices'', 
\emph{IEEE Transactions on Computers}, vol. 68, no. 10, 2019.

\bibitem{bge2004}
O. Billet, H. Gilbert, C. Ech-Chatbi, 
``Cryptanalysis of a white box AES implementation'', 
\emph{SAC 2004}, LNCS volume 3357, 2004.

\bibitem{xl2009}
Y. Xiao, X. Lai, 
``A secure implementation of white-box AES'', 
\emph{2nd International Conference on Computer Science and its Applications, IEEE CSA}, 2009.

\bibitem{mulder_xl2013}
Y. De Mulder, P. Roelse, B. Preneel, 
``Cryptanalysis of the Xiao–Lai white-box AES implementation'',
\emph{SAC 2013}, LNCS volume 7707, 2013.

\bibitem{xu2018}
T. Xu, C. K. Wu, F. Liu, R. Zhao, 
``Protecting white-box cryptographic implementations with obfuscated round boundaries'', 
\emph{Sci. China Inform. Sci.}, 61(3), 2018.

\bibitem{yeom2020}
Y. Yeom, DC. Kim, C. H. Baek, J. Shin, 
``Cryptanalysis of the Obfuscated Round Boundary Technique for Whitebox Cryptography'', 
\emph{Sci. China Inform. Sci.}, 63, 2020.

\bibitem{isobe2015}
A. Bogdanov and T. Isobe, 
``White-box cryptography revisited: Space-hard ciphers'', 
\emph{ACM SIGSAG Conference on Computer and Communications Security}, ACM, 2015.

\bibitem{cho_wem2017}
J. Cho, Y. Choi, I. Dinur, O. Dunkelman, N. Keller, D. Moon, A. Veidberg, 
``WEM: A New Family of White-Box Black Ciphers Based on the Even-Mansour Construction'', 
\emph{CT-RSA 2017}, LNCS volume 10159, 2017.

\bibitem{baek2016}
 C. H. Baek, J. H. Cheon, H. Hong, 
 ``White-box AES implementation revisited'', 
 \emph{Journal of Communications and Networks}, 2016.
 
\bibitem{biryukov2003}
A. Biryukov, C. De Canniere, A. Braeken, B. Preneel, 
``A toolbox for cryptanalysis: Linear and affine equivalence algorithms'', 
\emph{EUROCRYPT 2003}. LNCS volume 2656, 2003.

\bibitem{biryukov2014}
A. Biryukov, C. Bouillaguet, D. Khovratovich, 
``Cryptographic schemes based on the ASASA structure: Black-box, white-box, and public-key'', 
\emph{ASIACRYPT 2014}, LNCS volume 8873, 2014.

\bibitem{dinur2018}
I. Dinur, 
``An Improved Affine Equivalence Algorithm for Random Permutations'', 
\emph{EUROCRYPT 2018}. LNCS volume 10820, 2018.

\bibitem{dinur2015}
I. Dinur, O Dunkelman, T Karnz, G Leander, 
``Decomposing the ASASA Block Cipher Construction'', IACR Cryptol, 2015.

\bibitem{biryukov2015}
A. Biryukov, D. Khovratovich, 
``Decomposition attack on SASASASAS,'' 
\emph{Cryptology ePrint Archive}, Report 2015/646, 2015.

%=====



\end{thebibliography}

\EOD

\end{document}

\begin{IEEEbiography}[{\includegraphics[width=1in,height=1.25in,clip,keepaspectratio]{Yongjin_Yeom.png}}]{First A. Author} (M'76--SM'81--F'87) and all authors may include 
biographies. Biographies are often not included in conference-related
papers. This author became a Member (M) of IEEE in 1976, a Senior
Member (SM) in 1981, and a Fellow (F) in 1987. The first paragraph may
contain a place and/or date of birth (list place, then date). Next,
the author's educational background is listed. The degrees should be
listed with type of degree in what field, which institution, city,
state, and country, and year the degree was earned. The author's major
field of study should be lower-cased. 

The second paragraph uses the pronoun of the person (he or she) and not the 
author's last name. It lists military and work experience, including summer 
and fellowship jobs. Job titles are capitalized. The current job must have a 
location; previous positions may be listed 
without one. Information concerning previous publications may be included. 
Try not to list more than three books or published articles. The format for 
listing publishers of a book within the biography is: title of book 
(publisher name, year) similar to a reference. Current and previous research 
interests end the paragraph. The third paragraph begins with the author's 
title and last name (e.g., Dr.\ Smith, Prof.\ Jones, Mr.\ Kajor, Ms.\ Hunter). 
List any memberships in professional societies other than the IEEE. Finally, 
list any awards and work for IEEE committees and publications. If a 
photograph is provided, it should be of good quality, and 
professional-looking. Following are two examples of an author's biography.
\end{IEEEbiography}

\begin{IEEEbiography}[{\includegraphics[width=1in,height=1.25in,clip,keepaspectratio]{Yongjin_Yeom.png}}]{Second B. Author} was born in Greenwich Village, New York, NY, USA in 
1977. He received the B.S. and M.S. degrees in aerospace engineering from 
the University of Virginia, Charlottesville, in 2001 and the Ph.D. degree in 
mechanical engineering from Drexel University, Philadelphia, PA, in 2008.

From 2001 to 2004, he was a Research Assistant with the Princeton Plasma 
Physics Laboratory. Since 2009, he has been an Assistant Professor with the 
Mechanical Engineering Department, Texas A{\&}M University, College Station. 
He is the author of three books, more than 150 articles, and more than 70 
inventions. His research interests include high-pressure and high-density 
nonthermal plasma discharge processes and applications, microscale plasma 
discharges, discharges in liquids, spectroscopic diagnostics, plasma 
propulsion, and innovation plasma applications. He is an Associate Editor of 
the journal \emph{Earth, Moon, Planets}, and holds two patents. 

Dr. Author was a recipient of the International Association of Geomagnetism 
and Aeronomy Young Scientist Award for Excellence in 2008, and the IEEE 
Electromagnetic Compatibility Society Best Symposium Paper Award in 2011. 
\end{IEEEbiography}

\begin{IEEEbiography}[{\includegraphics[width=1in,height=1.25in,clip,keepaspectratio]{Yongjin_Yeom.png}}]{Third C. Author, Jr.} (M'87) received the B.S. degree in mechanical 
engineering from National Chung Cheng University, Chiayi, Taiwan, in 2004 
and the M.S. degree in mechanical engineering from National Tsing Hua 
University, Hsinchu, Taiwan, in 2006. He is currently pursuing the Ph.D. 
degree in mechanical engineering at Texas A{\&}M University, College 
Station, TX, USA.

From 2008 to 2009, he was a Research Assistant with the Institute of 
Physics, Academia Sinica, Tapei, Taiwan. His research interest includes the 
development of surface processing and biological/medical treatment 
techniques using nonthermal atmospheric pressure plasmas, fundamental study 
of plasma sources, and fabrication of micro- or nanostructured surfaces. 

Mr. Author's awards and honors include the Frew Fellowship (Australian 
Academy of Science), the I. I. Rabi Prize (APS), the European Frequency and 
Time Forum Award, the Carl Zeiss Research Award, the William F. Meggers 
Award and the Adolph Lomb Medal (OSA).
\end{IEEEbiography}

\EOD

\end{document}
